%!TEX root = ../username.tex
\chapter{Introduction}\label{chapter:intro}

Within the world of digital signal processing (DSP), digital audio, and physical instruments, there are almost limitless possibilities available now to create sounds through computers, whether through synthesis, production, or mixing. Ultimately, the products which result from explorations between computer science and music are dependent on the creator's familiarity with both fields. Current applications for music synthesis are sufficient in usability, with Helm, an open-source modular synthesizer, as one example. One is easily able to find Helm through an Internet search, download the application, and being altering an input's sound to their liking, provided they have a MIDI controller on hand to serve as the input device. Thus, the primary goal of this thesis is to create a modular synthesizer with two options for inputs: MIDI and pure sound waveforms, such as sine waves. 

There are three goals that must be accomplished in order to provide a modular synthesizer which utilizes both MIDI and pure sound waveforms: synthesizing (creating) pure sound waveforms through our language of choice \textit{SuperCollider}, accepting MIDI as an input option, and altering either input properly so the sound modification is obvious. So, this project relies on knowledge of both digital signals (the detectable digital impulses through which messages or other information can be transmitted \cite{Rosen_Howell_2011}), for MIDI input, as well as waveforms types, and mathematically, the ways in which we are able to alter these signals. Creating a virtual modular synthesizer requires a significant amount of programming in the domain-specific language SuperCollider, which connects the input from a physical MIDI controller and pure sound waveforms to the desired sound modifications. Thus, there will be a front-end graphical user interface, as well as a back-end which is connected to the signal modification modules. The front-end will provide users with a way to easily interpret and understand the modules of this synthesizer, and the back-end will process the desired sound changes the user has input, and output these alterations. The back-end includes modules typically found in open-source and commercial modular synthesizers, including, but not limited to: volume control, pitch bend, harmonics layering, delay, legato and staccato, distortion, and manual MIDI adjustments. This virtual modular synthesizer successfully completes each of these goals, and will serve as the primary product of this thesis.

Several of these modules, or functionalities which are able to act independently from each other, are self-explanatory; the pitch bend will alter the perceived pitch (frequency) of the input note, and the volume control will adjust the sound's loudness. The other modules mentioned are less intuitive. Harmonics layering involves layering various frequencies, whether the same or slightly different, over one another. Delay affects how early or late a note is perceived to be played. Legato, as defined in music, is the playback of notes in a connected manner, and staccato is the opposite, the playback of notes in a detached and separated way. Distortion typically is a \textit{destructive} effect, in which the sound to be distorted cannot be reverted back to its previous state. For this modular synthesizer, distortion will act as another version of harmonics layering, adding unneeded and unpleasant sounds to a user's input. For MIDI input, there is also an option to manually adjust the sound output a user receives, through adjusting an \textit{ADSR envelope}, or the Attack, Decay, Sustain, and Release of a MIDI note.

First, we will provide an introduction to the field of DSP, sound synthesis, and MIDI, and SuperCollider, the domain-specific programming language used for this project. This discussion will include context on previous developments on modular synthesizers and define the difference between modular sound synthesis and other types of synthesis. The modular synthesizer, specifically, is a type of synthesizer, or electronic instrument, which is able to produce a wide variety of sounds, through sound modifications, in a single unit with a unified control system. Then, we discuss the important mathematical details which are relevant to the creation of this modular synthesizer. Specifically, this will include the various types of waveforms, and the ways in which the waves can be manipulated. After, we go into detail of the process of developing the modular synthesizer, and the step-by-step description of the ways in which each module was developed. Some modules proved to be more challenging to implement than others, and so we also provide context as to how a particular module is developed. Finally, we finish with a conclusion discussing the limitations of implementing this synthesizer, the ways in which this synthesizer could be expanded or improved, and suggestions for future work.