%!TEX root = ../username.tex
\chapter{Introduction}\label{chapter:intro}

\section[Motivation]{Motivation}\label{section:motivation}
signal processing
synthesizer

\input{chapters/project-goals}
% real-time audio
% post processed audio
% SuperCollider

\section[How this will work]{How this will work}
% Math
% SuperCollider

To create a virtual analog modular synthesizer, there are two goals that must be accomplished: converting a real-time MIDI signal (or a post-processed audio file) into a format which the program can read, and a synthesizer application which can read this input and output a user's desired sound. Mathematically, this requires an understanding of digital signals and sine waves. 

On the flip side, creating a virtual modular synthesizer for this project requires a significant amount of programming in the domain-specific language SuperCollider. This is to create both a user interface which will interact with the user, and back-end logic, which will do the computations required to manipulate the input audio wave.