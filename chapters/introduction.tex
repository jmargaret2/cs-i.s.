%!TEX root = ../username.tex
\chapter{Introduction}\label{chapter:intro}

\section[Motivation]{Motivation}\label{section:motivation}
signal processing
synthesizer

\section[Project Goals]{Project Goals}\label{project-goals}
The first goal of this project is to build a working prototype of a virtual modular synthesizer using the domain-specific language SuperCollider. This custom synthesizer will contain modules commonly found in other virtual or analog modular synthesizers, including a pitch bender, volume control, BPM control, a legato switch, an arpeggiator, and a delay effect. Other modules may be included, but this project will focus on including the previously mentioned modules. 

Several of these modules are self-explanatory, as the pitch bender will change the pitch (the frequency) of the note heard, and the volume control will affect the loudness of the sound. The other modules mentioned are less intuitive. While it may be easy to figure out that the bpm (beats per minute) module will change the velocity, or speed, of the audio output that we hear, the legato switch, arpeggiator, and delay effect may not be. In music, playing legato affects how the smoothness between notes sounds. So, the legato switch, or slider, will change the transition between notes that are outputted in the synthesizer. With the value of the switch higher, the transition between notes will sound smoother, and the opposite with a lower value of the switch. An arpeggio, also known as an arpeggiated chord, is defined as a chord which is sounded individually, rather than simultaneously. Each note in the chord will be sounded, one after the other, instead of all the notes sounding at once. The arpeggiator will mimic the creation of an arpeggiated chord, separating out several notes played in unison into their individual notes. These notes will then sound one by one in accordance to which note was pressed or sounded first. Finally, the delay effect involves delaying a sound for a certain time interval after it was originally sounded. Our prototyped custom synthesizer will then receive an input, and store the input in temporary memory for a fixed period of time. After this interval passes, the synthesizer will release the input to the synthesizer's output channels, where it will sound. This creates an echo-like effect, where the original audio is heard, then some time later the delayed audio is heard. 

The second goal of this project deals with the input process of the custom synthesizer itself. Many current synthesizers, whether virtual or analog, contain one or more input options. Inputs for this project will include a MIDI signal, and eventually an audio file. 
% real-time audio
% post processed audio
% SuperCollider

\section[How this will work]{How this will work}
% Math
% SuperCollider

To create a virtual analog modular synthesizer, there are two goals that must be accomplished: converting a real-time MIDI signal (or a post-processed audio file) into a format which the program can read, and a synthesizer application which can read this input and output a user's desired sound. Mathematically, this requires an understanding of digital signals and sine waves. 

On the flip side, creating a virtual modular synthesizer for this project requires a significant amount of programming in the domain-specific language SuperCollider. This is to create both a user interface which will interact with the user, and back-end logic, which will do the computations required to manipulate the input audio wave.