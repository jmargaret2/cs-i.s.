\section[Modular Synthesis: What Is It?]{Modular Synthesis: What Is It?}\label{modular-synth-what-is}

A modular synthesizer is a type of synthesize which is composed of separate modules for different functions. These modules, in a physical hardware synthesizer, would be connected together by the user to create a \say{patch}. The output from these modules would then be audio signals, voltages, or digital signals for various logical or timing conditions. Typically, these modules would include voltage-controlled oscillators, voltage-controlled filters, voltage-controlled amplifiers, and envelope generators.\footnote{The need for voltage-controlled modules was such that the modules would not receive sufficient power otherwise to properly function.} Within a virtual modular synthesizer, we have no need for these voltage-controlled modules. However, we still must \say{patch} these modules together in a linear sort of way, and decide the order in which the modules are applied to a digital signal. 