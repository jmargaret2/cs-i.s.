\section[Modular Synthesis: What Is It?]{Modular Synthesis: What Is It?}\label{modular-synth-what-is}

A modular synthesizer is a type of synthesize which is composed of separate modules for different functions. These modules, in a physical hardware synthesizer, would be connected together by the user to create a \say{patch}. The output from these modules would then be audio signals, voltages, or digital signals for various logical or timing conditions. Typically, these modules would include voltage-controlled oscillators, voltage-controlled filters, voltage-controlled amplifiers, and envelope generators.\footnote{The need for voltage-controlled modules was such that the modules would not receive sufficient power otherwise to properly function.} Within a virtual modular synthesizer, we have no need for these voltage-controlled modules. However, we still must \say{patch} these modules together in a linear sort of way, and decide the order in which the modules are applied to a digital signal. 

To do so, there is no set order. Virtual modules, unlike their physical counterparts, are easily able to be patched together in any order. Based on the order that the user changes the values of each module, the synthesizer itself will then change the audio output to align. It is user dependent on which module is triggered first. How the modular synthesizer makes these changes will be explained in the next chapter. 

Modular synthesis itself is very different from the other two major types of synthesis. The first, additive synthesis, involves starting with a rudimentary audio wave, typically a sine wave. Then, other sounds and harmonics are added to this wave, hence the name additive. After several additions, the result is a complex, composite audio wave, in which multiple sounds and harmonics sound in unison, or as close to simultaneously as possible. [ADD CITATION FOR ADDITIVE SYNTHESIS] The second is subtractive synthesis. This type of synthesis is the opposite of additive synthesis. Instead of starting with a baseline audio wave, we will start with a complex, composite audio wave. Then, aspects of this wave are stripped away, leaving only a fundamental harmonic sound. [ADD CITATION FOR SUBTRACTIVE SYNTHESIS] The audio wave in stripped of its extra sounds and layers, into its base sine wave, or whichever wave form makes up the sound. Modular synthesis is slightly different than these two. Instead of creating an entirely new sound, or stripping a sound of various aspects, it involves changing a pre-existing sound. Aspects of the sound, ranging from volume and pitch to more advanced concepts, are changed through changing how the audio wave itself will sound.