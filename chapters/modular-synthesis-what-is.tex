\section[Modular Synthesis: What Is It?]{Modular Synthesis: What Is It?}\label{section:modular-synth-what-is}

A modular synthesizer is a type of synthesizer which is composed of separate modules for different functions. These modules, in a physical hardware synthesizer, would be connected together by the user to create a \say{patch}. The output from these modules would then be audio signals, voltages, or digital signals for various logical or timing conditions. Typically, these modules would include voltage-controlled oscillators, voltage-controlled filters, voltage-controlled amplifiers, and envelope generators. There was a need for voltage-controlled modules, such that the module would not receive sufficient power otherwise in order to properly function. Within a virtual modular synthesizer, we have no need for these voltage-controlled modules. However, we still must \say{patch} these modules together in a linear sort of way, and decide the order in which the modules are applied to a digital signal. 

To do so, there is no set order. Virtual modules, unlike their physical counterparts, are easily able to be patched together in any order. Based on the order that the user changes the values of each module, the synthesizer itself will then change the audio output to align. It is user dependent on which module is triggered first.

As previously mentioned in sections \ref{section:project-goals} and \ref{section:how-it-works}, there are various types of modules used within synthesizers, such as oscillators, filters, amplifiers, mixers, envelope generators, sequencers, and much more. The basic modular functions involve effects to an audio wave's signal itself, the control of an audio wave, or the logic--or timing--of the audio signal\cite{Gabrielli_2020}. These modular functions can be categorized into one of two groups: a source module, or a processor module. A source module can be characterized by a certain out, but has no signal input value. There are multiple types of source modules, the main one being the oscillator (with the primary ones being either a voltage-controlled oscillator--VCO--or a low-frequency oscillator--LFO) which impact the sound that is generated from a synthesizer. The oscillator itself is a type of module which generates a repeating signal\cite{Gabrielli_2020}. The rise and fall of the signal is what gives the module its name, as the signal will oscillate between its trough and its peak\cite{Rosen_Howell_2011}. The speed with which the wave rises and falls, or reaches its peaks and troughs, is known as the wave's \textit{frequency}. If the frequency is not too low or too high, then the human ear will be able to hear it. The generally accepted range of human hearing is between 20 Hz and 20 kHz, and so frequency of audio waves will fall between these two values. The VCO will output a signal, where the signal's frequency is a function of the oscillator. In its most basic form, a VCO will be a simple waveform, usually a square or a sawtooth wave which can be changed dynamically through a control such as frequency modulation. The LFO will be operated using a period with a length of anywhere from a fortieth of a second up to several minutes. Generally, the LFO will be used as a control for a different module; with an oscillator, the LFO will produce a modulation to the signal's frequency , and with an amplifier, the LFO will produce a change in the signal's amplitude. Certain changes to a signal's frequency may produce an effect known as \textit{vibrato}. Vibrato is a musical effect in which there is a small regular and pulsating change to a signal's pitch. It is most used to add expression and phrasing to both vocal and instrumental music, and is characterized by two factors: the first, the amount the pitch is varied (the \say{extent of vibrato}), and the second, the speed at which the pitch is varied (or the \say{rate of vibrato}). Modulations to the signal's amplitude are different. These changes may produce \textit{tremolo}, or a trembling effect in music. There are two distinct types of tremolo: rapid reiteration, and one based on a variation in a signal's amplitude. Rapid reiteration will occur due to either a single note, two notes, two chords, or as a roll on a percussive instrument. As a single note, it is particularly noticeable on a bowed string instrument, such as the violin or viola. On these instruments, a single note may be played by rapidly moving the bow back and forth. On a non-bowed instrument, such as the guitar, tremolo can be produced by plucking a single note repeatedly. As tremolo between notes or chords, these will be played in alternation, switching between note or chord one, then quickly playing note or chord two, and back again. At times, the terms for tremolo and vibrato in non-classical music are used incorrectly or interchangeably. In the world of classical music, they are both properly defined as separate musical effects. Vibrato is defined as the periodic variation in a note or signals' pitch (frequency), while tremolo is defined as the fast repetition or the same note or signal to produce the effect of a longer note being sounded. In practice, however, it is difficult for a performer to achieve pure vibrato or pure tremolo, as in that case only a note's pitch (frequency) or volume (amplitude) would be allowed to be varied. So, variations to both pitch and volume will often be made at the same time. In the world of popular music, this means that the definitions for both terms will be modified slightly; vibrato will retain its same meaning as it does in classical music, with a a periodic variation in a note or signal's pitch, but tremolo will instead refer to a periodic variation in volume, achieved by other electric effects.
Processor modules are different, and characterized by having both a signal input and an output. These modules include the filter, amplifier, low-pass gate, mixers, limiters, and more. The first type, the filter, in its most basic sense will remove (or filter out) frequencies from an audio signal, as desired by the user. The most common types of filters are the low-pass filter (LP) and high-pass filter (HP). Both, as evidenced by their name, will filter out frequencies of a certain band, and let other frequencies pass. Low-pass filters will filter out high frequencies, allowing low frequencies to pass, while high-pass frequencies will filter out low frequencies, letting high frequencies pass\cite{Winer_2018}. Other less common filter types include the bandpass (BP) filter, and all-pass filters, on certain synthesizers. The bandpass filter will attenuate both low and high frequencies, allowing only a certain range of frequencies around a specified cutoff point to pass through. The point at which a filter will begin working is known as the cutoff point, which is set by the module's cutoff knob itself. Filters will be used to alter the timbre, or the tone of the signal's sound. Another type of processor module is the amplifier. An amplifier is a component of a modular synthesizer which will change the amount of a signal which passes through the module. 
As referenced in section \ref{section:mod-synth-history}, there are four types of synthesis: modular synthesis, what this paper will focus on, wavetable synthesis, additive synthesis, and subtractive synthesis. A wavetable is a collection of single-cycle waveforms, or samples of audio, which are played on a loop to produce a periodic waveform, such as a sine wave. Then, wavetable synthesis will use a table, which contains the values of several frequencies played in a certain order (the wavetable). When a note is pressed, or a MIDI note on command is given, the signal will move through the table in the order specified, smoothly changing the shape of the signal into the various waves specified on the table. This produces an output which can evolve both quickly and smoothly, as it will follow the table of waveforms to modify the input signal. It will typically offer the widest range of sound that could be created or modified, in comparison to other techniques of sound synthesis\cite{Gabrielli_2020}.
Subtractive synthesis begins with a complex, composite audio wave. The input signal is stripped of its extra sounds and layers, as it goes through a chain of modules. Some harmonics present in the input signal will be reduced into harmonic structures which mimic the harmonics of actual instruments, or will be reduced to a fundamental harmonic sound, into its base waveform\cite{Gabrielli_2020}. The modules associated with subtractive synthesis involve filters and envelop adjustments. The filter's \say{cutoff frequency} or \say{cutoff point} changes depending on the type of filter being used. The main filter types used in subtractive synthesis include the low-pass filter, the high-pass filter, the band-pass filter, and the notch, or band-reject filter. The two new filter types to us are the band-pass filter, and the notch filter. The band-pass filter will only let a select range of frequencies pass through it, with its cutoff frequency as the center point for this range. The notch filter will do the opposite, removing a select range of frequencies around a certain cutoff point\cite{Gabrielli_2020}.

Additive synthesis is the opposite of subtractive synthesis. We begin with a rudimentary audio wave or waveform, typically a sine wave. Then, sounds, frequencies, or harmonics are added on top of this fundamental audio waveform, creating a new timbre\cite{Winer_2018}. It attempts to achieve the same goal as subtractive synthesis, only through constructive measures, rather than destructive. 

As previously mentioned, this project will focus on including six distinct modules for the prototyped modular synthesizer: a pitch bender, volume control, BPM control, legato switch, arpeggiator, and a delay effect. Each of these modules will be a processor-type module. They will contain a signal input (the MIDI note a user plays), and output modulations to this note, based on the module selected. The order in which these modules are used is entirely user-dependent, and does not need to fall into a set linear order. 