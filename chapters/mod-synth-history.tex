\section[History of Modular Synthesizers]{History of Modular Synthesizers}\label{mod-synth-history}

The 1950s was a turning point within the field of electronic music and communicational technologies. This marked the creation of electronic music, with the use of physical componenets. Electronic music was rare, with synthesizers\footnote{As described in section \ref{modular-synth-what-is}.} and electronic instruments\footnote{Examples of these would include electric guitars, electric keyboards, and more.} being less common. So also were modular synthesizers, at that point containing fewer modules\footnote{The concept of modules will be further discussed in section \ref{modular-synth-what-is}.} than found in modular synthesizers today. These modular synthesizers were large, taking up significant space, as evidenced by Bode's Audio System Synthesizer and Moog's Minimoog Model D, two influential analog modular synthesizers discussed later in this section.

The first module, or functionality, created was the positive feedback oscillator, sometime between 1912 and 1914 \cite{Gabrielli_2020}.\footnote{This module is now commonly found in modular synthesizers today, both analog (physical) and digital (virtual).} A positive feedback oscillator--or as it is known today, the simple oscillator--produces sounds within a synthesizer, specifically a repeated signal (the audio wave) which periodically rises and falls. This is where the oscillator module gets its name, producing a sound which rises and falls, or oscillates.\footnote{The method through which the oscillator creates the audio waveform affects the resulting waveform, of which examples are discussed in chapter \ref{theory}. For now, it is important to note that oscillators will produce one of four waveforms typically: sine waves, triangle waves, sawtooth waves, and square waves, or pulse waves.} The creation of this positive feedback module is attributed to Austrian engineer and physicist Alexander Meissner. Unlike the oscillator module found today, the early version was only able to last a few minutes. Instead of continuously able to produce an audio waveform, this version produced only a small output of power \cite{Fleming_1919}. 1912 introduced the communicational technologies field as something more than niche. Products from proper physics laboratories were rare to find, and so were machines capable of producing audio waveforms with high frequencies after several minutes. Meissner created a new type of machine through the use of feedback: a high-frequency generator from an amplifier, with a vacuum tube as a substitute for a high-frequency receiver machine \cite{Fleming_1919}. Previous to Meissner's creation, it was impossible to establish a connection between a high-frequency machine with a receiver. The output frequency of such a machine was not the same as the user's desired output frequency, a synchronous and superimposed frequency. Meissner's product was easier to handle than these machines, and much more useful. In addition, Meissner's machine was much less noisy as well. This machine could change the output frequency given to the receiver, and tune the receiver machine to a range of wavelengths that were previously not possible \cite{Fleming_1919}. This development changed the process of receiving wavelengths. The un-dampened wave\footnote{An un-dampened audio wave is a typical audio wave. The wave continues to oscillate without any force or opposing motion to stop it from oscillating. This differs from a dampened wave. A dampened wave is an audio wave which oscillates, yet there is some force or opposing motion which slows the wave down. Eventually, the wave will lose energy such that it stops oscillating, unless there is a force or motion to keep the wave oscillating.} which resulted allowed for a user to fine-tune the frequency, while also suppressing any atmospheric distractions or sounds. With Meissner's work, the field's knowledge of constant sound generation and high frequencies was expanded, and paved the way for future work on the short-wave band \footnote{Though not discussed in this paper, the short-wave band is typically used in maritime communications, international, and radio broadcasting. More reading on Alexander Meissner and his patented solution can be found in the Engineer and Technology History Wiki's page on Alexander Meissner here: \url{https://ethw.org/Alexander_Meissner}}.

Then, sometime between 1959 and 1960, Harold Bode (1909-1987), a German engineer, created the modular synthesizer which we are familiar with today. Bode was a German engineer and designer of audio tools. He foresaw that transistor technology\footnote{A transistor is a semiconductor device which is used to amplify, control, and generate electrical signals. These are active components for integrated circuits, or \say{microchips}. Transistors are tiny, and so these microchips will often contain billions of transistors, etched into their surfaces. Today, transistors have become embedded into almost everything electronic.} would become a key change in creating and designing synthesizers, especially modular synthesizers\cite{Gabrielli_2020}. Transistors would link the audio signal through cables, through where each component, or \say{module} could be connected in any order, according to the user's preferences. Multiple modules--including modulators, filters, reverberation generators, and more--could then be connected in any order, either to modify or generate sounds. With transistor technology, Bode created his \textit{Audio System Synthesizer}, which allowed for a larger number of sound creation possibilities than before. The system itself contained inputs for various sound sources, and the input signals could be modified by filter modules and a modulator\cite{Bode_1984}. As the system was modular, there were independently working modules for sound modification. These modules could be combined with each other in several ways, according to the user's desired order. This Audio System Synthesizer also made an impression on Robert Moog, who would take Bode's idea and further develop it \cite{Gabrielli_2020}. 

Robert Moog (1934-2005) was an American engineer inspired by Harold Bode's Audio System Synthesizer. As the inventor of the first commercial synthesizer, dubbed the Moog synthesizer, Moog created the first integrated synthesizer. The Moog synthesizer was built in 1964, and contained several of the fundamental synthesizer concepts found today\cite{Pinch_Trocco_2004}. These modules on the Moog synthesizer included the voltage-controlled modules, envelope generators\footnote{Envelope generators will be dicussed in more depth in section \ref{modular-synth-what-is}}, and the pitch wheel. With the Moog synthesizer, synthesizers were brought to a wider audience, influencing the development of popular music\cite{Pinch_Trocco_2004}. What followed the Moog synthesizer was the Minimoog Model D, a portable 1970 creation. The Minimoog Model D is acknowledged to be the most \say{influence synthesizer of all time}\cite{Gabrielli_2020} due to its playability and compactness. Similar to the Moog synthesizer, it included modules to both generate and shape sounds \cite{Pinch_Trocco_2002}.

The 1960s also saw the introduction of Donald "Don" Buchla (1937-2016) and the Buchla 100 Series Modular Electric Music System. Today, Buchla is regarded as a pioneer in electronic musical instrument design.

Buchla's ``West Coast" method is less conventional, due to the use of dual-linked oscillators to create complicated waveforms processes through low-pass gates or filters. Until recently, modular synthesizers were never popular, but are experiencing a resurgence due to the acceptance of a de facto standard: Eurorack. By 2017, over 100 companies, including Moog, were developing Eurorack modules.