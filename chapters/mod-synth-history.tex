\section[History of Modular Synthesizers]{History of Modular Synthesizers}\label{mod-synth-history}

The 1950s was a turning point, as it marked the creation of electronic music. Electronic music was rare, and so synthesizers and electronic instruments were less commonplace. Modular synthesizers also contained fewer modules than those found in modern-day modular synthesizers. The first module was the positive feedback oscillator, a module commonly found in synthesizers now, was invented sometime between 1912 and 1914 \cite{Gabrielli_2020}. This is best attributed to Austrian engineer and physicist Alexander Meissner. The early version of the oscillator was only able to last for a few minutes' use at a time, producing a small output of power \cite{Fleming_1919}. Before 1912, communications technologies was a niche field. Products from physics laboratories were few and far in between. This was especially noticeable in machines with high frequencies. It was technically impossible to establish a connection between a high-frequency machine to a receiver. The output frequency would not be the same as the desired synchronous and superimposed multiplied frequency. Meissner solved this issue through the use of feedback, creating a high-frequency generator from an amplifier, and using a vacuum tube as a substitute for a high-frequency machine. This final product Meissner created was much more useful, and easier to handle, than previous noisier machines. Meissner's product could change the frequency in the receiver, and tune the receiver in a range of wavelengths that were previously not possible. This development changed the process of receiving wavelengths, as using the undampened wave [INSERT DEFINITION OF UNDAMPENED WAVE] allowed for fine-tuning of the frequency, and suppress some atmospheric distractions. With the knowledge that this device can generator more constant vibrations, and very high frequencies, Meissner paved the way for work on the short-wave band [INSERT DEFINITION]\footnote{More reading on Alexander Meissner and his patented solution can be found in the Engineer and Technology History Wiki's page on Alexander Meissner here: \url{https://ethw.org/Alexander_Meissner}}.

Later, German engineer Harold Bode (1909-1987) invented the modular synthesizer we know today. Bode, a German engineer and designer of audio tools, was designing electronic musical instruments as early as 1937. He foresaw that transistor technology would be a key change in designing synthesizers. This new technology would have the signal linked together by cables, where each component could be connected in any order that the user wanted. Multiple devices (or "modules") could be connected--including modulators, filters, reverb generators, and more--in any order to modify or generate sounds. Known as Bode's Audio System Synthesizer, it allowed for a higher number of sound creation possibilities. This Audio System Synthesizer also made an impression on Robert Moog, who would take Bode's idea and further develop it \cite{Gabrielli_2020}. 

Robert Moog was one engineer who was inspired by Bode's Audio System Synthesizer. In the 1960s, Robert (Bob) Moog assembled the first integrated synthesizer built on subtractive audio \cite{Gabrielli_2020}\footnote{A singular filter can process a complex, composite sound in many ways that ultimately remove selected frequencies \cite{Winer_2018}}. His Minimoog Model D is acknowledged to be the first and most influential synthesizer of all time due to its playability and compactness \cite{Gabrielli_2020}. Built in 1959, this synthesizer included separate modules to create and shape sounds, such as an envelope follower, envelope generator, amplifier, filters, mixers, pitch extractors, and more\footnote{Further reading can be found in Pinch and Trocco, 2002}, much like Bode's modular synthesizer.

The 1960s also saw the introduction of Donald "Don" Buchla (1937-2016) and the Buchla 100 Series Modular Electric Music System. Today, Buchla is regarded as a pioneer in electronic musical instrument design.

Buchla's ``West Coast" method is less conventional, due to the use of dual-linked oscillators to create complicated waveforms processes through low-pass gates or filters. Until recently, modular synthesizers were never popular, but are experiencing a resurgence due to the acceptance of a de facto standard: Eurorack. By 2017, over 100 companies, including Moog, were developing Eurorack modules.