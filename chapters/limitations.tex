\section{Limitations}\label{section:limitations}

The development of a virtual modular synthesizer has achieved its goals, but limitations in the process of implementation still exist. The first of these, and the most noticeable in the development of this synthesizer, involves the scope of functions in SuperCollider. SuperCollider has two types of scope for functions and variables: local, and environmental (similar to how many other languages function). If a variable is declared within a certain scope, like a function, the variable will thus have a local value only within that scope, as variable \texttt{a} does in Listing \ref{lst:good-sc-code}. This code lies between parenthetical brackets (simply known as ``brackets'' in SuperCollider) which determine the scope of a variable. However, lowercase letters (a-z, with the exception of `s' which by default is used as a reference to the \textit{scsynth} server) are able to be used with declaration, as are ``environmental'' variables, with contain the \texttt{\~} symbol \cite{McCartney_2016}. While good that some variables are able to be accessed outside a particular scope, this is not good software development. Only single lowercase letters can be global, and if we were to use longer variable names, we would use environmental variables, declared with the \texttt{\~} symbol, and are seen as global variables within SuperCollider. So for the purposes of developing a modular synthesizer, we avoid the usage of global variables to store the necessary information for each module. This requires us to place each module into the same scope, nested into only one set of parenthetical brackets, rather than multiple. We lose the ability to create different functions in different files or classes of code, as the modules must maintain their modularity, or their ability to function in any order, and affect each other.