\chapter[Modular Synthesizer]{Building a Virtual Analog Modular Synthesizer}\label{chapter:building-a-mod-synth}

%https://www.gemtracks.com/guides/view.php?title=music-genres-and-their-typical-bpms&id=823 -> cite for SuperCollider tempo module
%http://hyperphysics.phy-astr.gsu.edu/hbase/Music/mussca.html -> for arpeggiator module, on how to decide how to dynamically get the third and fifth intervals -> also need to explain interval ratios

\section{Building the GUI}\label{section:building-the-gui}

\section{The Modules: Pure Waveforms}

The waveforms created through SuperCollider are through the ``SinOsc'' Unit Generator. As in Listing \ref{lst:sinewave-synthdef}, a simple sine wave is created, with a frequency of 440 Hz, and a volume of 50. Thus, as previously seen in both Figure \ref{fig:basic-sine-wave} and Equation \ref{eq:sine-wave-equation}, a continuous sine wave has been created.

\begin{listing}
	\begin{lstlisting}
		SynthDef("sinewave", {arg freq=440, vol=50; Out.ar(0, SinOsc.ar(freq, 0, vol))}).add;
	\end{lstlisting}
	\caption{Creating a sine wave SynthDef in SuperCollider}
	\label{lst:sinewave-synthdef}
\end{listing}

\begin{listing}
	\begin{lstlisting}
		x = Synth("sinewave");
	\end{lstlisting}
	\caption{Putting the sine wave SynthDef into a Synth, for sound output}
	\label{lst:sinewave-synth}
\end{listing}

\subsection{Volume Slider}

Volume is a simple concept to understand, as it is how loud or soft the human ear hears at a particular frequency. Volume ranges, as they are known in classical music, begin at a soft, and at times, barely audible \textit{pianissimo}, and become as loud as \textit{fortissimo}. The sine wave SynthDef created in Listing \ref{lst:sinewave-synthdef} contains a variable value for volume. In the SynthDef, the volume is initialized at the number 50, which generally correlates to a medium-loud volume of \textit{mezzo forte}. The variable A of the generic sine wave equation \ref{eq:sine-wave-equation} is equivalent to this change in volume.

As the sine wave Synth Def contains two variables, one for frequency, and one for volume, creating modules for both a volume slider and a pitch knob is simple. To create the volume slider, a SuperCollider class called \textit{EZSlider} creates the outline of the volume slider itself, as in Figure \ref{fig:volume-slider-basic}. For the slider's functionality, there are three important parts: the ``controlSpec,'' ``action,'' and ``initVal.'' The ``controlSpec'' defines the ``control spec,'' or the range of values allowed for the specified module. Negative volume does not exist, so this simple volume module will contain valid values for 0 volume (\textit{pianissimo}) up to 100 volume (\textit{fortissimo}). Then, the ``action'' argument of the \textit{EZSlider} class determines the function that runs when the value of the volume slider is changed. 

\begin{figure}[h]
  \centering
  \includegraphics[width=\textwidth]{volume-slider-basic.png}
  \caption{The basic volume slider, with a volume of 50, or \textit{mezzo forte}}
  \label{fig:volume-slider-basic}
\end{figure}

\begin{listing}
	\begin{lstlisting}
		volumeSlider = EZSlider(awindow, label:"Volume", controlSpec:[0,100], action:{|mv| x.set("vol", mv.value)}, initVal:50);
	\end{lstlisting}
	\caption{Creating the volume slider in SuperCollider}
	\label{lst:volume-slider-waveform}
\end{listing}

In Listing \ref{lst:volume-slider-waveform}, the action that is set involves changing the volume of the Synth created in Listing \ref{lst:sinewave-synth}. In this code example, the SynthDef from Listing \ref{lst:sinewave-synthdef} that is assigned to the variable name ``sinewave'' is now put into a Synth. This Synth, as previously described, allows SuperCollider to deal with audio output. So, the SynthDef ``sinewave'' is put into a Synth, known as ``x.'' Within the action, $x$ is able to be manipulated, that is the sine wave which is in the Synth can be manipulated, resulting in an altered sound. The action itself begins with a reference to the volume slider, $mv$. The Synth, $x$, then references the volume argument from the SynthDef, ``vol,'' which sets the volume for both the SynthDef and the Synth, and uses the \texttt{set} function to set the volume of the Synth equal to the volume that the volume slider contains. The volume of the Synth $x$ will update as the value of the volume slider does, setting the value of \texttt{x.vol} to be equivalent to \texttt{mv.value}. Finally, ``initVal'' simply initializes the starting value of the slider to volume 50 (\textit{mezzo forte}).

\subsection{Pitch Knob: The Pitch Bend}

Pitch, as previously mentioned, is simply a functionality which dictates the frequency of a note that the human ear perceives. Standard Western tuning currently dictates notes to be tuned around the starting pitch of the note A above Middle C (or $A_5$), which is equivalent to 440 Hz. In Listing \ref{lst:sinewave-synthdef}, we notice that the sine wave SynthDef is created, with two arguments: the frequency for the sine wave to play at, and the volume of the sine wave. The frequency of the sine wave that is created is equivalent to the variable \textit{B} in the generic sine wave equation (Equation \ref{eq:sine-wave-equation}).

Similar to the volume slider, creating the pitch knob in SuperCollider relies on the use of a native class, \textit{EZKnob}. This class creates the knob, as in Figure \ref{fig:pitch-wheel-basic}. Like with the class \textit{EZSlider}, the knob also has the ``controlSpec,'' ``action,'' and ``initVal.'' The controlSpec for the pitch knob is \texttt{freq}, denoting the span of available frequencies, of the valid notes within standard tuning systems. While negative frequency values do not exist, valid frequency values would exist within the human range of hearing (that is, 20 Hz to 20 kHz). The ``action'' argument works similarly to its functionality in the volume slider, as it sets the \texttt{freq} argument of SynthDef ``sinewave'' equal to the frequency value of the pitch knob. Then, the Synth of Listing \ref{lst:sinewave-synth}, ``x,'' continuously matches the value of the pitch knob (\texttt{mn.value}) to the frequency value of the SynthDef, and thus also the Synth. The final argument of the \textit{EZKnob} class is ``initVal,'' in which the initial value of the pitch knob is set to 440 Hz.

\begin{figure}
  \centering
  \includegraphics{pitch-wheel-basic.jpg}
  \caption{The basic pitch wheel, with a pitch of 440 Hz, or $A_5$}
  \label{fig:pitch-wheel-basic}
\end{figure}

\begin{listing}
	\begin{lstlisting}
		pitchKnob = EZKnob(awindow, label:"Pitch", controlSpec:\freq, action:{|mn| x.set("freq", mn.value)}, initVal:440);
	\end{lstlisting}
	\caption{Creating the pitch knob in SuperCollider}
	\label{lst:pitch-knob-waveform}
\end{listing}



\subsection{Legato Switch: A Sustain Button}

The \textit{legato} switch for this synthesizer is meant to emulate the \textit{legato} notation found in classical music. \textit{Legato} is a directive, typically found in its full form in classical music, which indicates the performance of a specific passage to be played in a smooth, graceful, and connected style (opposed to the \textit{staccato} notation) \cite{Winer_2018}. It will often be indicated by a slur over the notes, or an accent mark with a line over the notes to be affected, as in Figure \ref{fig:legato-example}\cite{Henle_2009}. On a physical electronic keyboard, this module is most often seen with a \textit{sustain} button, in which the notes played are extended, and slurred into each other. However, it is important to note that not all slur lines in written sheet music will be meant to be played \textit{legato}. Notes that are to be played \textit{legato} will differ in pitch, connecting notes of different pitches to be played in succession in a smooth manner. This is unlike the notation for tied notes, as in Figure \ref{fig:tied-notes-example}\cite{Lung_2016}, which connect notes that are the same pitch. The durations of the notes which are tied together are combined, and played at that new longer note duration instead. However, further development on this module is not needed. A continuous sine wave was created through the SynthDef, and later put into a Synth for sound output. A module which sustains a note, or in the case a waveform, smoothly connects it to the next waveform and will not alter the sound of the continuous waveforms that are created within SuperCollider code.

\begin{figure}[h]
  \centering
  \includegraphics[width=\textwidth]{bartok-dance-four-b-section-second-system.jpg}
  \caption{An example of \textit{legato} notation, found in Bela Bartok's \textit{Six Romanian Folk Dances}, Sz. 56, BB 68}
  \label{fig:legato-notes-example}
\end{figure}

\begin{figure}[h]
  \centering
  \includegraphics[width=\textwidth]{bach-prelude-second-motive.jpg}
  \caption{An example of tied notes, found in J.S. Bach's \textit{Prelude in C Minor}, from \textit{The Well-Tempered Clavier, Book I}, BVW 847}
  \label{fig:tied-notes-example}
\end{figure}

\subsection{Major Chord Generator}

To understand the major chord generator, we must first understand what a major chord is. A chord can be defined as the simultaneous sounding of two or more notes (typically three or more). Most chords are triadic in nature (that is, containing only three notes), with the interval of a major third or minor third between each of the three notes. The major third interval can be defined as the interval which spans three degrees of the diatonic scale in the Western twelve-semitone tuning system (refer to subsection \ref{subsection:how-midi}), or four semitones.\footnote{The major third interval is also enharmonically equivalent to the diminished fourth interval.} The minor third interval contains one fewer degrees than the major third interval, thus having only two degrees of the diatonic scale, and so only three semitones. For instance, the interval between $A$ and $C\musSharp{}$ is a major third, as the note $C\musSharp{}$ lies four semitones away from the note $A$, while the interval between $A$ and $C$ is a minor third, with the note $C$ lying only three semitones away from the note $A$. Notable examples of the major third interval include the first two notes of the song ``When the Saints Go Marching In,'' the first movement of Ludwig van Beethoven's \textit{Fifth Symphony} (Figure \ref{fig:beethoven-fifth}\cite{Beethoven_1862}), or the song ``Swing Low, Sweet Chariot.'' Examples of the minor third interval include the first two notes of the tune of ``Greensleeves,'' (Figure \ref{fig:greensleeves}\cite{Kurtz_2010}) Christmas tune ``What Child is This,'' or The Beatles' ``Hey, Jude.''
\begin{figure}
  \centering
  \includegraphics[width=0.5\textwidth]{beethoven-fifth.jpg}
  \caption{An example of a major third interval, found in the First Movement of Ludwig van Beethoven's \textit{Symphony No. 5 in C Minor}, Op. 67}
  \label{fig:beethoven-fifth}
\end{figure}

\begin{figure}
  \centering
  \includegraphics[width=0.4\textwidth]{greensleeves.jpg}
  \caption{An example of a minor third interval, found in the English Folk Song ``Greensleeves''}
  \label{fig:greensleeves}
\end{figure}


The interval of the third is important to distinguish \textit{major} chords, and \textit{minor} chords, as major chords will have a root note (the tonic note), major third interval, and another minor third interval (or perfect fifth interval above the tonic) stacked on top of one another, while a minor chord will have the tonic, minor third, and perfect fifth.

The Major Chord Generator within this modular synthesizer involve two pieces: calculating the proper frequencies of the major third and perfect fifth intervals--as they are subject to change from the user's input on the pitch knob--and adding these two intervals to play simultaneously with the original waveform sound. First, to calculate the proper intervals of tonic note to major third, and tonic note to perfect fifth, \textit{interval ratios} (the widths of semitones) are used. As mentioned within Subsection \ref{subsection:how-midi}, the commonly-used tuning system in the Western world is the twelve-tone equal temperament system, which divides the octave into 12 parts. Each of these parts are equally tempered (or equally spaced) on a logarithmic scale, such that each ratio is equal to $2^\frac{1}{12}$, or $\sqrt[12]{2} \approx 1.05946$ (the 12th square root of 2). This tuning system is normally tuned relative to the standard pitch of 440 Hz, known as \textit{A440}, signifying that the note $A$ (typically $A_4$) is tuned to 440 Hertz, and all other notes are defined relative to this pitch, as some multiple of semitones away from it, either higher or lower in frequency. Thus,  the modular synthesizer created also begins with a starting pitch of A440, every other pitch is defined relative to it. 

Within the twelve-tone equal temperament system, calculating the intervals for the major third and perfect fifth is some multiplication of the single semitone $\sqrt[12]{2}$. As there are four semitones between the tonic note and its major third, the interval ratio is $2^\frac{4}{12}$, or $\sqrt[\frac{4}{12}]{2}$, and seven semitones between the tonic and perfect fifth, the interval ratio is $2^\frac{7}{12}$, or $\sqrt[\frac{7}{12}]{2} \approx \sqrt[12]{128}$, as in Listing \ref{lst:chord-creation}. %TODO: fix this it doesn't make sense

\begin{listing}
	\begin{lstlisting}
		var baseFrequency = 440; // A4 -> synthesizer is based A = 440 is standard tuning
		var thirdFreq = baseFrequency**(4/12);
		var fifthFreq = baseFrequency** (7/12);
	\end{lstlisting}
	\label{lst:chord-creation}
	\caption{Creating the major third and perfect fifth intervals}
\end{listing}

After calculating the ratio of frequencies, the work on the second part of the module begins. Similar to the work done to create the initial waveform, two additional pure waveforms, and their Synth counterparts, are created (Listing \ref{lst:major-chord-synths}). Then, the only step which remains involves adding the Synths $y$ and $z$ into a variable to play simultaneously (Listing \ref{lst:major-chord-module}) using an array called ``majChord.''

\begin{listing}
	\begin{lstlisting}
		SynthDef("sinewave_third", {arg vol=50; Out.ar(0, SinOsc.ar(thirdFreq, 0, vol))}).add;
		SynthDef("sinewave_fifth", {arg vol=50; Out.ar(0, SinOsc.ar(fifthFreq, 0, vol))}).add;
		y = Synth("sinewave_third");
		z = Synth("sinewave_fifth");
	\end{lstlisting}
	\label{lst:major-chord-synths}
	\caption{Creating SynthDefs for the major third and perfect fifth intervals}	
\end{listing}

\begin{listing}
	\begin{lstlisting}
		majChord = ["sinewave", "sinewave_third", "sinewave_fifth"];
	\end{lstlisting}
	\label{lst:major-chord-module}
	\caption{Combining the three waveform Synths into an array ``majChord''}
\end{listing}

The ``majChord'' variable is placed into a button class \textit{Button}, which has two states, depending on the on/off status of the button itself (Listing \ref{lst:major-chord-button}). Once the button's status is changed to on, all three Synths within the ``majChord'' array will sound at once.

\begin{listing}
	\begin{lstlisting}
		majorChord = Button(awindow, Rect(20, 20, 150, 25)).states_([["Turn Major Chord Off", Color.black, Color.gray], ["Turn Major Chord On", Color.black, Color.yellow]]);	
	\end{lstlisting}
	\label{lst:major-chord-button}
	\caption{Implementing the major chord module using the \textit{Button} class}
\end{listing}

\subsection{Delay Slider}

The ``delay'' effect (or ``echo'' effect as it is sometimes known) is an effect created by a tool, such as this project's modular synthesizer, which records an input signal, stores it, then plays it back after a defined time. Typically, the delayed audio is mixed with the live audio input creating an echo effect, where we first hear the original audio, followed by the delayed audio. To create this module, we must think back to the unit circle (Figure \ref{fig:unit-circle}), and the generic equation for a sine wave (Equation \ref{eq:sine-wave-equation}). The value of \textit{D} in Equation \ref{eq:sine-wave-equation} determines the shift of the wave, and thus also the sound. A positive value (such as $\frac{\pi}{6}$) will shift the sine wave to the left on the Cartesian plane, resulting in a sine wave which sounds early. The same applies to a negative \textit{D} value, in which a value such as $-\frac{\pi}{6}$ shifts the sine wave to the right, creating a sine wave which sounds ``late'' or delayed.

To create the delay slider, which determines the time which the waveform is delayed, we rely on the same native SuperCollider class we did to create the volume slider: \textit{EZSlider}. Like with the volume slider, we are using the same three arguments: controlSpec, action, and initVal. For the control spec of the delay slider, we must make sure these values match those available in the unit circle, so that the valid range of values is between $\frac{\pi}{6}$ to $\frac{11\pi}{6}$. While it is possible to use values greater than $\frac{11\pi}{6}$, or less than $\frac{\pi}{6}$ in theory, in practice it will sound equivalent to values within this range, as the sine wave will be overlaid directly on top of the possible values in this range. 

\begin{listing}
	\begin{lstlisting}
		delaySlider = EZSlider(awindow, label:"Delay Time", labelHeight:50, labelWidth:100, controlSpec:[(-pi)/6, pi/6], action: {|md| x.set("phase", md.value)}, initVal:0);	
	\end{lstlisting}
	\caption{Creating a delay slider in SuperCollider}
	\label{lst:delay-slider}
\end{listing}

In SuperCollider the time values for delay are calculated in radians, so using the values from the unit circle works well. For the ``action'' argument, similarly to previous modules, the phase of Synth ``x'' is set to be equivalent to the value of the delay slider. The slider itself is initialized to 0, where there is no early or late arrival of the sound.


\section{MIDI Input}\label{section:midi-input}

Input with MIDI in SuperCollider is more complex than creating pure sound waveforms for modular changes, however there are some similarities; we must create three aspects for modular sound synthesis: the \texttt{Synth}, \texttt{SynthDef}, and MIDI functions for MIDI itself to be able to output sound. 

Before any of this, however, the MIDI inputs and the MIDI client must be initialized. As in Listing \ref{lst:initialize-midi}, there are two key steps to using MIDI in SuperCollider: initializing the MIDI client and connecting to the specific MIDI controller that will be used. Once this is done, we can move on to creating the MIDI functionality itself.

\begin{listing}
	\begin{lstlisting}
		MIDIClient.init;
		MIDIIn.connect;
	\end{lstlisting}
	\caption{Initializing the MIDI Client}
	\label{lst:initialize-midi}
\end{listing}

Creating a SynthDef for MIDI input will be the easiest task. To create a SynthDef for MIDI use, we do what is in Listing \ref{lst:midi-synthdef}. 

\begin{listing}
	\begin{lstlisting}
		// A SynthDef with an ADSR envelope
		SynthDef("piano", {arg freq = 440, amp = 0.1, gate ;
		var snd, env;
		env = Env.adsr(0.01, 0.1, 0.3, 2, amp).kr(2, gate);
		snd = Saw.ar(freq: [freq, freq*1.5], mul: env);
		Out.ar(0, snd)
		}).add;
	\end{lstlisting}
	\caption{Creating a MIDI SynthDef with a ADSR envelope}
	\label{lst:midi-synthdef}	
\end{listing}

Listing \ref{lst:midi-synthdef} describes the necessary aspect in the creation of a SynthDef that will  be compatible with MIDI commands. The ADSR envelope, or ``envelope generator'' (a term which refers to the ``shape'' of a sound, or the contour by which a sound gets louder and softer) is described by its stages: attack, decay, sustain, release. Some of this was discussed in the Note On and Note Off messages subsections of Section \ref{section:midi-messages}. The order in which sound goes through an envelope generator is important, as sound must travel through the attack, decay, sustain, and release stages in that order, and is unable to go back to any other stage once it comes to a stage. An ADSR envelope generator will first receive a gate input, the variable \texttt{gate}, and raise the value of \texttt{gate} to the maximum volume or voltage (if we were to use a physical envelope generator within an analog synthesizer) the envelope generator is able to output, or the output level that is set over a specific time by the Attack control. The gate is one of the main signal types of a modular synthesizer. It will jump from a base level (normally 0) to a higher one when a new note is meant to start, such as when a user presses down on a MIDI keyboard, or another transition is meant to happen, such as when the next stage of the ADSR envelope is meant to start. When a user presses down on a MIDI keyboard, the gate will typically stay at the level it was given for the duration of that note, while the note or key is being held down, and then suddenly drop to its baseline level once the key is released. Thus, when the \texttt{gate} is sent through a typical envelope generator like the ADSR envelope generator, the beginning of the gate increases to its maximum volume as it tells the envelope to go through the Attack and Decay stages. As the gate remains at this high level, the envelope may go into the Sustain level, and then when the gate's volume returns to its baseline level, the envelope will move into the Release stage. 

Once the sound reaches this defined high level of the Attack stage, the Decay control will cause the sound to begin dropping in volume, until it reaches the volume set by the Sustain control. If \texttt{gate} is still active (and has a value other than 0), the level set by the Sustain control is maintained until the value of \texttt{gate} returns to 0--which typically signifies the user has released the key on a compatible keyboard or controller. When \texttt{gate} is no longer active, the output volume begins to drop back to a volume of 0, in which the rate of this drop is determined by the Release control. 

A key concept to understanding the ADSR envelope is that there is a difference in behavior when the ADSR envelope is not able to finish the entire four-stage cycle. If the user were to release the key before the Attack or Decay stages finish, then the envelope may skip to the Release stage, passing over the Sustain control entirely, and continuing to the Release control with the current volume level. However, if the user were to re-trigger the envelope by sending a new \texttt{gate} through to the synthesizer, a digital envelope generator will return the volume level back to 0, and restart the envelope cycle, while an analog envelope generator will start the cycle from the current voltage level.

Other arguments within the SynthDef of Listing \ref{lst:midi-synthdef} include \texttt{freq}, and \texttt{amp}, which will help in modifying the input MIDI key presses. Creating the Synth and MIDI functions for proper MIDI functionality are much more difficult. When creating the waveform modules, there was no need to develop functionality for the waveforms, as they were a native aspect of SuperCollider. Within the MIDI modules, creating the Synth and MIDI functions must be done simultaneously. As in Listing \ref{lst:midi-note-on-and-off}, the two primary MIDI messages that will be used are note on and note off. To create a Synth using the ``piano'' SyntheDef from Listing \ref{lst:instantiate-synthedef}, it will be wrapped within the SuperCollider MIDI function for note on: \texttt{MIDIdef.noteOn}. With the proper MIDI functionality created, similar modules to those developed for pure waveforms can be made.

\begin{listing}
	\begin{lstlisting}
		on = MIDIdef.noteOn(\myKeyDown, {arg vel, note;
				notesArray[note] = Synth("piano", [\freq, note.midicps, \amp, vel.linlin(0, 127, 0, 1)]);
				x = Synth("piano", [\freq, note.midicps, \amp, vel.linlin(0, 127, 0, 1)]);
				["NOTE ON", note].postln;
			});

		off = MIDIdef.noteOff(\myKeyUp, {arg vel, note;
			notesArray[note].set(\gate, 0);
			["NOTE OFF", note].postln;
		});
	\end{lstlisting}
	\caption{Creating MIDI note on and MIDI note off messages}
	\label{lst:midi-note-on-and-off}	
\end{listing}


\section{The Modules: In Midi}\label{section:the-modules-midi}
% how the MIDI input was changed

\subsection{Pitch Bend}