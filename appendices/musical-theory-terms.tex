\chapter{Music Theory Terms}\label{appendix:music-theory-terms}
The following list contains the definitions of frequently used musical terms in this paper. Additionally, terms which may augment the understanding of the changes made musically are included. 
\begin{itemize}
    \item {\textbf{Frequency}: the perceived pitch of a sound}
    \item {\textbf{Volume}: the perceived loudness of a sound}
    \item {\textbf{Timbre}: the quality of a sound, which helps to differentiate between instruments}
    \item {\textbf{Staccato}: a directive for notes to be played detached and separated}
    \item {\textbf{Legato}: a directive for notes to be played smoothly and connected}
    \item {\textbf{Chord}: the simultaneous sounding of two or more notes. Typically, a chord will be composed of three notes in total, created a \textit{triad}.}
    \item {\textbf{Major chord}: a chord composed of a root note (the tonic note), a major third interval above the tonic note, and a perfect fifth interval above the tonic note.}
    \item {\textbf{Tonic note}: the note of a chord or a song which determines the key signature.}
    \item {\textbf{Key signature}: a set of sharp (\musSharp{}, or flat (\musFlat{}) symbols placed on the staff at the beginning of sheet music or a section of music.}
    \item {\textbf{Major third interval}: the interval that spans four semitones. For example, the interval between $C$ and $E$ is a major third.}
    \item {\textbf{Minor third interval}: the interval that spans three semitones. For instance, the interval between $A$ and $C$ is a minor third.}
    \item {\textbf{Texture}: how a sound is organized, and the number of layers within a sound.}
    \item {\textbf{Treble clef}: a type of musical notation to indicate the pitches represented by the lines and spaces on sheet music. Also known as the ``G-clef,'' the second line from the bottom represents the note \textit{G} above Middle C. This clef is the most common clef seen. Typically, the treble clef will contain the note Middle C, as well as notes above Middle C.}
    \item {\textbf{Alto clef}: a type of musical notation to indicate the pitches represented by the lines and spaces on sheet music. This clef is also known as the ``C-clef'' or the ``Viola clef,'' as only certain instruments, which include the viola, use this clef. The middle line of this clef represents the note Middle C.}
    \item {\textbf{Pianissimo}: a directive to perform an indicated passage of a composition or piece very softly. Abbreviated as \textit{pp}.}
    \item {\textbf{Fortissimo}: a directive to perform an indicated passage of a composition or piece very loudly. Abbreviated as \textit{ff}.}
\end{itemize}