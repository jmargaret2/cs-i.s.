\chapter{Music Theory Terms}\label{appendix:music-theory-terms}
The following list contains the definitions of frequently used musical terms in this paper. Additionally, terms which may augment the understanding of the changes made musically are included. 
\begin{itemize}
    \item {\textbf{Frequency}: the perceived pitch of a sound.}
    \item {\textbf{Volume}: the perceived loudness of a sound.}
    \item {\textbf{Timbre}: the quality of a sound, which helps to differentiate between instruments.}
    \item {\textbf{Staccato}: a directive for notes to be played detached and separated.}
    \item {\textbf{Legato}: a directive for notes to be played smoothly and connected.}
    \item {\textbf{Tie/tied notes}: for this directive, a curved line is drawn over or under the heads of notes of the same pitch. This indicates that there should be no break in the playing of these notes, and should be played as one singular note.}
    \item {\textbf{Chord}: the simultaneous sounding of two or more notes. Typically, a chord will be composed of three notes in total, created a \textit{triad}.}
    \item {\textbf{Major chord}: a chord composed of a root note (the tonic note), a major third interval above the tonic note, and a perfect fifth interval above the tonic note.}
    \item {\textbf{Tonic note}: the root note of a chord or a song, which determines the key signature.}
    \item {\textbf{Key signature}: a set of sharp (\musSharp{}), or flat (\musFlat{}) symbols placed on the staff at the beginning of sheet music or a section of music.}
    \item {\textbf{Major third interval}: the interval that spans four semitones. For example, the interval between $C$ and $E$ is a major third.}
    \item {\textbf{Minor third interval}: the interval that spans three semitones. For instance, the interval between $A$ and $C$ is a minor third.}
    \item {\textbf{Texture}: how a sound is organized, and the number of layers within a sound.}
    \item {\textbf{Treble clef}: a type of musical notation to indicate the pitches represented by the lines and spaces on sheet music. Also known as the ``G-clef,'' the second line from the bottom represents the note \textit{G} above Middle C. This clef is the most common clef seen. Typically, the treble clef will contain the note Middle C, as well as notes above Middle C.}
    \item {\textbf{Alto clef}: a type of musical notation to indicate the pitches represented by the lines and spaces on sheet music. This clef is also known as the ``C-clef'' or the ``Viola clef,'' as only certain instruments, which include the viola, use this clef. The middle line of this clef represents the note Middle C.}
    \item {\textbf{Pianissimo}: a directive to perform an indicated passage of a composition or piece very softly. Abbreviated as \textit{pp}.}
    \item {\textbf{Fortissimo}: a directive to perform an indicated passage of a composition or piece very loudly. Abbreviated as \textit{ff}.}
    \item {\textbf{Interval ratio}: the ratios of the frequencies of pitches in a musical interval. As an interval is the ``distance'' between two pitches, the ratio assists musicians to work with relative pitch measures applicable to a range of instruments intuitively, rather than a set of memorized frequency values. A simpler ratio will sound more pleasing to the ear, and thus more consonant, than complex ratios.}
    
    For instance, suppose we have a guitar as in Figure \ref{fig:guitar-math}. The \textit{interval ratio} will then be inverse to the length of the string. The total length of the string in red has a 1:1 ratio, and the remaining pitches can be described as some ratio to this total string length. On the E string of this guitar (the top string of Figure \ref{fig:guitar-math}), the note an octave above the note E is still E. Then, this note E one octave above the root note E is 12 frets above the root. As noted in the Figure, pressing down on fret 12 of the E string (or any string) results in the length of the string being halved, and an interval ratio of $\frac{1}{2}$. This produces the note one octave above the starting note.
    \item {\textbf{Equal temperament}: a system of tuning the scale, in which the octave is evenly divided into 12 equal semitones. It is based on the cycle of 12 identical fifths, or the ``circle of fifths'' \cite{Cochrane_2011}.}
    
    \begin{figure}[H]
        \centering
        \includegraphics[width=0.4\textwidth]{figures/guitar-math.jpeg}
        \caption{Pythagoras Ratios for Guitar Frets}\cite{Passy_2012}
        \label{fig:guitar-math}
    \end{figure}
    \item {\textbf{Tritone interval}: the most dissonant interval in the diatonic scale. This interval spans six semitones. For instance, the interval between C and F\musSharp{} is a tritone.}
    
    In the Medieval era of music, the tritone was said to be the ``devil's interval'' because it was the most dissonant (unpleasant) interval in the diatonic scale. Its unpleasant sound to the human ear can be traced back to a phenomenon found within the human brain. The human brain is hardwired to find harmony and symmetry within music, and so enjoys consonant, pleasing sounds, and is resistant towards the less-pleasing dissonant sounds. 
    
    Intervals which sound pleasing to the human ear are those in which there is a simple ratio between intervals (and thus also frequencies). As described in the definition of \textbf{interval ratio}, certain interval ratios are designed to sound more pleasing than others. Each note of the scale, from a root note, contains a particular frequency interval ratio. This frequency ratio, as we ascend the scale, oscillates between consonant and dissonant intervals. The tritone, as the most dissonant of all intervals on the diatonic scale, contains a large frequency interval ratio of 45:32 (or 64:45, depending on the tuning method).
    \item {\textbf{Diatonic scale}: the scale which we have frequently discussed in this paper. It is the scale which contains, from a root tonic note, five whole tones and two semitones. Both the typically used major and natural minor scales are diatonic, with the two semitones falling between the third and fourth tones, as well as the seventh and eighth tones in the major scale. The natural minor scale is slightly different, as the semitones will fall between the third and fourth tones, and the fifth and sixth tones. From a root tonic note, each of these tones a whole tone and a semitone above can also be described in ``scale degrees'' (which has the same meaning as whole tone and semitone). Both the major and natural minor scales help to create the key signature, which is part of the foundation of music.}
    
    For example, in the key of C Major, the two semitone intervals will fall between the third and fourth scale degree, as well as the seventh and eighth scale degree. This results in a semitone difference between the notes E and F, and B and C. In the key of A Minor, the semitone intervals fall between the third and fourth scale degrees (C and D), and the fifth and sixth scale degrees (E and F).
    
\end{itemize}