\chapter{The Frequency Ranges Within the Human Range of Hearing}
\begin{table}
	\begin{tabular}{|p{20em} | p{25em}|}
		\hline
		General Frequency Range & Description of Range \\ 
		\hline
		<20Hz - 60Hz & The lowest threshold of human hearing. This includes many frequencies that are felt and not heard, and provides the \say{rumble} feeling in music. This range gives much of music its power, and is typically known as \say{sub-bass}. \\
		\hline
		60Hz - 250Hz & This range determines the amount of \say{warmth} and how full the sound is perceived to be. The notes fundamental to rhythm lives in this range, and too much sound in this frequency range will result in the overall sound being \say{boomy}, or muddy-sounding and messy. It is otherwise known as the the \say{bass} frequency. \\
		\hline
		250Hz - 500Hz & The lower harmonics of many instruments are in this range. It is generally known as the \say{lower midrange} of frequencies, and can introduce listening fatigue and a telephone-quality to the sound if this range is emphasized too much. \\
		\hline
		500Hz - 2 kHz & This range is considered the middle of the midrange. It gives many instruments prominence in a mix, and determines how audible one instrument or vocalist is in comparison to another. If this range is emphasized, audio output may sound tinny and small, which could lead to listening and ear fatigue, as the human ear is sensitive to the human voice, and the frequencies it covers. \\
		\hline
		2 kHz - 4 kHz & The \say{upper midrange} is responsible for much of the attack sounds on percussive and rhythmic instruments. This range may add presence to the mix if boosted, but if it is emphasized too much, it may mask some speech recognition sounds. Listening fatigue may also set in if this range is emphasized too much, as the slightest boost in this range will result in a noticeable change in the sound's timbre. \\
		\hline
		4 kHz - 6 kHz & This range is known as the \say{presence} range. It defines a sound's clarity and the definition of voices and instruments that are present. If this range is boosted, instruments and voices may sound physically closer to the listener, and vice versa, with reducing this range causing instruments and voices to sound further away. However, if this range is emphasized too much, a harsh, irritating sound may occur. \\
		\hline
		6 kHz - 20 kHz & This range controls the \say{brilliance} and clarity of sounds within the mix. Instead of pitches, this range is composed entirely of harmonics, and brings \say{sparkle} to the sound. This range also may easily cause ear fatigue, as an over-emphasis can increase the hiss heard, and produce sibilance, or an unpleasant tonal harnshness which can happen with consonant syllables (most noticeably: S, T, and Z), especially on vocals. \\
		\hline
	\end{tabular}
\caption{The general frequencies ranges, within the range of human hearing}\cite{Suits_1998}\cite{Zjalic_2021}
\label{tbl:frequency-table-of-human-hearing-general}
\end{table}